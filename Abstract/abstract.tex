

\begin{abstracts}    

Trong những năm gần đây, ngành công nghệ thông tin đang phát triển rất rất mạnh mẽ và nhanh chóng. Tiêu biểu  là AI - Artificial Intelligence (Trí tuệ nhân tạo), và cụ thể hơn nữa là Machine Learning (Máy Học) nổi lên như  một bằng chứng rõ rệt của một cuộc cách mạng công nghiệp lần thứ tư sắp xảy ra (1 - động cơ hơi nước, 2 - điện năng, 3 - công nghệ thông tin). Trí tuệ nhân tạo đang len lỏi vào mọi lĩnh vực trong đời sống mà có thể rất nhiều người trong chúng ta không nhận ra. Hệ thống nhận diện tự tag khuôn mặt của Facebook \footnote{https://www.facebook.com/notes/facebook/making-photo-tagging-easier/467145887130/}, hệ thống gợi ý sản phẩm của Amazon \footnote{https://www.amazon.com/}, xe tự hành của Google và Tesla, trợ lý ảo Siri của Apple\footnote{https://www.apple.com/ios/siri/}, hệ thống gợi ý phim của Netflix \footnote{https://piay.iflix.com}, và mới đây, với một sự kiện chấn động, tháng 5 năm 2016 máy chơi cờ vây AlphaGo \footnote{https://deepmind.com/research/alphago/} của Google DeepMind đã chiến thắng cả con người (nhà vô địch cờ vây thế giới Lee Se-dol) trong một trò chơi mà số khả năng biến hóa có thể xảy ra là $10^{761}$ nhiều hơn tất cả số lượng nguyên tử trong vũ trụ \par
	Khi mà công nghệ ngày càng phát triển, khả năng tính toán của các máy tính được nâng lên một tầm cao mới cộng với một lượng dữ liệu khổng lồ được các hãng công nghệ lớn thu thập. Machine Learning đã tiến thêm một bước tiến dài và một lĩnh vực mới ra đời có tên là Deep learning (Học Sâu). Deep Learnig đã giúp cho máy tính thực thi được những công việc tưởng chừng như không thể vào khoảng 10 năm trước đây như: phân loại đối tượng khác nhau trong một bức ảnh, Giao tiếp với con người, hay thậm chí là sáng tác âm nhạc, văn thơ. \par 
	Nhận thấy tính ứng dụng cao, sự hiệu quả và tiềm năng phát triển của Deep Learning sau này. Sinh viên chọn đề tài "Nghiên cứu áp dụng Deep Learning cho bài toán đếm đối tượng trong ảnh tĩnh" với tính ứng dụng cao và có thể áp dụng cho nhiều lĩnh vực khác nhau trong thực tế. Qua đề tài này, sinh viên muốn tìm hiểu về Deep Learning và áp dụng kiến thức đã tìm hiểu được để xây dựng một phương pháp đếm các đối tượng (con người, phương tiện) tự động từ hình ảnh của người dùng cung cấp.  \par 


\end{abstracts}
 